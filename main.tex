
\documentclass[bachelor]{thesis-uestc}
\title{论文标题}{Title of the Thesis}

\author{宋邦睿}{Bangrui Song}
\advisor{名字\chinesespace }{name}
\school{计算机科学与工程(网络空间安全)学院}{School of Computer Science and Technology (School of Cyber Security)}
\major{计算机科学与技术}{Computer Science and Technology}
\studentnumber{2023110903001}

% 设置图像路径为 ./pic/ 文件夹
\graphicspath{{./pic/}}

% require all the usepackages here
% \usepackage{algorithm2e}

\begin{document}

\makecover

% This is a template of mutiple files.
% The folders chapters/ and misc/ have the related files

% abstract
\begin{chineseabstract}
    音乐艺术的起源,历来有几种重要的学说。具有代表性的学说有模仿说、游戏说 、巫求说 、心灵表理说、劳动说等。在西方音乐艺术的发展史中,声乐艺术占据重要地位。其深厚的历史底蕴可追溯至古希腊和古罗马时期,这一
    时期不仅孕育了许多杰出的作曲家,也见证了声乐艺术形式在日常生活中的广泛融合。进入中世纪,声乐艺术得到了更广泛
    的传承与发展,标志着这三个历史时期声乐艺术源远流长,直至今日依然在不断流传,而且人们对音乐艺术有了更深刻的认
    识和诠释。
    
    \chinesekeyword{声乐艺术;艺术起源;中西方艺术}
\end{chineseabstract}

% table of contents
\thesistableofcontents

% thesis contents
% \input{chapter/chapter_1}
% \input{chapter/chapter_2}
% \input{chapter/chapter_3}
% \input{chapter/chapter_4}


% miscellaneous
% \input{miscellaneous/acknowledgement}
% \input{miscellaneous/appendix}

%==================reference================%
% 用uestc-thesis环境定义的参考文献格式导入参考文献
% % Uncomment to list all the entries of the database.
% % \nocite{*}
% \thesisbibliography{reference}

% %原生的`\bibliography`命令导入文献列表
% %Uncomment the following code to load bibliography database with native
% \bibliography command.
% %用于生成完整的参考文献列表,不管在正文中是否有实际引用。
% \nocite{*}

% \bibliographystyle{thesis-uestc}
% \bibliography{reference}
%

%外文期刊原文&译文
% \thesisaccomplish{publications}w
% \input{miscellaneous/translate_original}
% \input{miscellaneous/translate_chinese}

\end{document}
